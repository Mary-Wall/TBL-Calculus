\documentclass{exam}

\usepackage{amsmath,amssymb,amsthm}

\title{Math 121 - Readiness Assessment \# 6}
\date{}
%Key - D030

\begin{document}
\maketitle

\begin{questions}
\question The equation $\ln (x) = y$ really means what?
\begin{choices}
\choice $e^x = y$
\choice $e^y = x$
\choice $ex = y$
\choice $ey = x$
\end{choices}
\question The derivative of the natural logarithm, $\frac{d}{dx} \ln(x)$ is
\begin{choices}
\choice $e^x$
\choice $\ln(x)$
\choice $x$
\choice $\frac{1}{x}$
\end{choices}
\question What exponential function doesn't change when you take it's derivative?
\begin{choices}
\choice $a^x$
\choice $e^x$
\choice $x^a$
\choice $x^e$
\end{choices}
\question If we're given the function $f(x)$ and it's inverse $f^{-1}(x)$, which of the following is not true?
\begin{choices}
\choice $f^{-1}(f(x)) = x$
\choice $f^{-1}(x)$ is a one-to-one function.
\choice $f^{-1}(x) = \frac{1}{f(x)}$
\choice $f(f^{-1}(x)) = x$
\end{choices}
\question The equation $\log_{5} (x) = y$ really means what?
\begin{choices}
\choice $5^x = y$
\choice $5x = y$
\choice $5y = x$
\choice $5^y = x$
\end{choices}
\question What is the antiderivative of $e^x$?
\begin{choices}
\choice $e^x$
\choice $\ln(x)$
\choice $xe^{x+1}$
\choice $\frac{1}{x}$
\end{choices}
\newpage
\question Which of these functions would be a good candidate to try logarithmic differentiation?\\
\begin{choices}
\choice $x^2$
\choice $x^x$
\choice $2^x$
\choice $2x$
\end{choices}
\question If $\sin^{-1}(x) = y$, what else is also true?\\
\begin{choices} 
\choice $\frac{1}{\sin(x)} = y$
\choice $\sin(x) = y$
\choice $\sin(y) = x$
\choice $\sin^{-1}(x) = \cos(x)$
\end{choices} 
\question Which of these is \textbf{NOT} a type of indeterminate form?\\ 
\begin{oneparchoices}
\choice $\frac{0}{0}$
\choice $\frac{\infty}{\infty}$
\choice $\infty - \infty$
\choice $\frac{1}{\infty}$
\end{oneparchoices}
\question L'Hopital's rule allows us to do what when evaluating a limit with indeterminate form?
\begin{choices}
\choice Write it as a quotient, and differentiate the numerator and denominator separately, then take the limit.
\choice Apply the quotient rule.
\choice Differentiate the whole function and take the limit of the new function.
\choice Change the value of the input into the limit to avoid division by zero.
\end{choices}\end{questions}
\end{document}