\documentclass{exam}

\usepackage{amsmath,amssymb,amsthm}

\title{Math 121 - Readiness Assessment \# 1}
%Key - D025
\date{}

\begin{document}
\maketitle

\begin{questions}
\question
Three of these might be descriptions of tangent lines. One is definitely not. Pick the one that can't be a tangent line.
\begin{choices}
\choice A line touching a curve exactly once.
\choice A line intersecting a curve twice.
\choice A line intersecting a circle twice.
\choice A horizontal line touching a circle. 
\end{choices}
\question
Suppose that $\lim_{x \rightarrow a} f(x) = L$. Only one of these can still be true when a limit exists, the other 3 would imply that the limit does not exist. Which one can still be true?
\begin{choices}
\choice If $f(a)$ does not exist.
\choice If $f(x)$ is not defined near $a$.
\choice If $f(x)$ isn't a function.
\choice If $L$ is only approached by $f(x)$ on one side.
\end{choices}
\question
Which of the following means the limit $\lim_{x \rightarrow a} f(x) = L$ will exist?
\begin{choices}
\choice When the left-hand limit exists, and the function is not defined to the right of $a$.
\choice When the right-hand limit exists, and the function is not defined to the left of $a$.
\choice When the left-hand and right-hand limits agree.
\choice When the limit from the left get very large, and the limit from the right gets very negative.
\end{choices}
\question
Suppose $\lim_{x\rightarrow a^+} f(x) = 1$ and $\lim_{x\rightarrow a^-} f(x) = 2$. Which of these is definitely true.
\begin{choices}
\choice The function is continous at $x=a$.
\choice The limit $\lim_{x \rightarrow a} f(x)$ does not exist.
\choice The function stops existing at $x=a$
\choice Kevin Bacon
\end{choices}
\question
Why can't we use the Direct Substitution Property (just plugging in the value) in evaluating this limit? $$\lim_{x\rightarrow 2} \frac{x^3 - 7x + 6}{x^2 -4}$$
\begin{choices}
\choice The denominator evaluates to 0.
\choice The function is rational.
\choice The numerator evaluates to 0.
\choice The function is not a polynomial.
\end{choices}
\newpage
\question Which of these is the definition of function that's continuous at a point $x=a$?
\begin{choices}
\choice A function that can be drawn without lifting the pencil.
\choice The function has a limit at all points.
\choice A function without any asymptotes.
\choice $\lim_{x \rightarrow a} f(x) = f(a)$
\end{choices}
\question
Which of the following functions is not continuous on all real numbers?
\begin{choices}
\choice $g(x) = x^2$
\choice $f(x) = \frac{1}{x}$
\choice $h(x) = \sin(x)$
\choice $j(x) = 5x^6 - 4x + 1$
\end{choices}
\question
What hypothesis is needed to use the Intermediate Value Theorem?
\begin{choices}
\choice The function must have a limit at all points on the interval $[a,b]$.
\choice The function must cross the $x$-axis.
\choice The function must be continuous on the closed interval $[a,b]$.
\choice If the function is $f(x)$, then $f(a) = f(b)$.
\end{choices}
\question
Which of the following best describes what occurs when we say $$\lim_{x \rightarrow \infty} f(x) = L?$$
\begin{choices}
\choice If $x = \infty$, then $f(x) = L$.
\choice If $f(x)$ is close to $L$, then $x$ must be very large.
\choice There is no value for which $f(x) = L$.
\choice By taking $x$ sufficiently large, we can make $f(x)$ as close to $L$ as we like.
\end{choices}
\question Which is the definition of the derivative of the function $f(x)$?
\begin{choices}
\choice $$f'(x) = \lim_{x \rightarrow 0} \frac{f(x+h) + f(x)}{h}$$
\choice $$f'(x) = \lim_{h \rightarrow 0} \frac{f(x+h) + f(x)}{h}$$
\choice $$f'(x) = \lim_{x \rightarrow 0} \frac{f(x+h) - f(x)}{h}$$
\choice $$f'(x) = \lim_{h \rightarrow 0} \frac{f(x+h) - f(x)}{h}$$
\end{choices}
\end{questions}
\end{document}