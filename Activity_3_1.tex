  \documentclass{article}
\usepackage[utf8]{inputenc}
\usepackage{fancyhdr}
\usepackage[margin=1.0in]{geometry}
\usepackage{amsmath}
\usepackage{changepage}
\usepackage{graphicx}
\usepackage[inline]{enumitem}
\usepackage{enumitem}
\pagestyle{fancy}
\chead{Math 121 - Applying Differentiation Techniques}

\begin{document}
\begin{enumerate}
\item How do we find the equation of the tangent line to a curve at a point?
\begin{itemize}
\item First, make sure the team recalls the point-slope formula for the equation of a line. We should be able to find the equation of a straight line using only the slope, and some arbitrary point on that line.
\item Given the curve $y = x^2$, what is the equation of the tangent line when $x = 3$?
\item Given the curve $f(x) = \sin(x) - \cos(x)$, find the equation of the tangent line at the point (0,-1).
\end{itemize} 

\item How do we linearly approximate functions with a derivative? 
\begin{itemize}
\item Suppose we want to find the approximate value of $\sqrt[3]{8.03}$.
\item Can we find the exact value of $\sqrt[3]{8}$ without a calculator? 
\item Find the equation of the tangent line to the curve $y = \sqrt[3]{x}$ at the when $x = 8$.
\item Find the value of the point on this tangent line when $x = 1.03$. Compare this to the result on your calculator for finding $\sqrt[3]{1.03}$.
\end{itemize}

\item How could we use this idea to find an approximate value for $\sqrt[5]{31.94}$?

\item Can we use the tangent line to help approximate a zero of a function?
\begin{itemize}
    \item Start with the equation $x^3 + 2x - 1 = 0$.
    \item It's tough to find an exact solution to this. First, show that the function $f(x) = x^3 + 2x - 1$ has at least one real zero on the interval $[0,1]$.
    \item Find the equation of the tangent line to the function $f(x)$ when $x = 1$. We'll need some notation, so let $x_0 = 1$
    \item Where does that tangent line cross the $x$-axis? Label that point $x_1$.
    \item Now find the equation of the tangent line to the curve when $x = x_1$. Where does that line cross the $x$-axis? Label that point $x_2$.
    \item If you were to keep doing this, make a conjecture about what would happen with the sequence of points $x_0, x_1, x_2, \ldots$.
\end{itemize}
\end{enumerate}

\end{document}