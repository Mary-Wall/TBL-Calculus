\documentclass{exam}

\usepackage{amsmath,amssymb,amsthm,graphicx}

\title{Math 121 - Final Standards Assessment}
\date{}

\begin{document}
\maketitle
\bigskip

  \fbox{\parbox{6in}{
    \vspace{10pt}
    \textbf{\large Your Name:}
    \vspace{10pt}
  }}
\begin{center}
\large
    For any problem you skip, place a line through the problem number. For any problem you want graded, circle the problem number and the objective number next to it.
\end{center}
\normalsize

\begin{enumerate}
\item Evaluate a limit using the limit laws.\\
%$$\lim_{x \rightarrow 1} \frac{x^2 - 2x+1}{x-1}$$
%$$\lim_{h \rightarrow 0} \frac{h^2 - 4h}{h}$$
%$$\lim_{x \rightarrow 2}  \frac{|x-2|}{x-2}$$
%$$\lim_{x \rightarrow -2} \frac{x^2 + 4x + 4}{x+2}$$
%$$\lim_{x \rightarrow 2}  \frac{x+2}{x-2}$$
%$$\lim_{x \rightarrow 0} \frac{|x|}{x}$$
$$\lim_{x \rightarrow 5}  \frac{x^2 - 5x}{x-5}$$
$$\lim_{x \rightarrow -3} \frac{x+3}{x^2 + 5x + 6}$$
\item Evaluate a limit at infinity.\\
%$$\lim_{x\rightarrow \infty} \frac{7x^2 + 5x - 1/3}{5x^3 +43}$$
%$$\lim_{x\rightarrow \infty} \frac{4x^5 - x^4 + 89x}{x^5 - 999}$$
%$$\lim_{x\rightarrow \infty} \frac{4x^2 + 3x - 7}{9x^2 + 4x - 9}$$
%$$\lim_{x\rightarrow \infty} \frac{8x^2 + 5}{x^4 - 5x + 6}$$
%$$\lim_{x\rightarrow \infty} \frac{6x^3 + 7x - 1}{5x^2 + 14x - 1}$$
%$$\lim_{x\rightarrow \infty} \frac{6x^4 + 5}{3x^4 - 5x + 6}$$
$$\lim_{x\rightarrow \infty} \frac{6x^3 + 6}{5x^4 + - 1}$$
$$\lim_{x\rightarrow \infty} \frac{6x^4 + +3x -2}{3x^2 + 6}$$
\item Evaluate an infinite limit.\\
%$$\lim_{x \rightarrow 1}  \frac{x}{2x-2}$$
%$$\lim_{x \rightarrow 0-} \frac{1}{x}$$
%$$\lim_{x \rightarrow 5}  \frac{1}{x-5}$$
%$$\lim_{x \rightarrow -2} \frac{3}{2+x}$$
%$$\lim_{x \rightarrow \frac{\pi}{2}} |\tan(x)|$$
%$$\lim_{x \rightarrow 0+} \frac{1}{x}$$
$$\lim_{x \rightarrow 6} \frac{x+2}{x^2-4x-12}$$
$$\lim_{x \rightarrow -1} \frac{x+1}{x^2-1}$$
\item Use limits to determine the horizontal and vertical asymptotes of a function.\\
%$$f(x) = \frac{2x^2 -4}{x^2 + 1}$$
%$$g(x) = \frac{2x^2 - x}{3x^2 - 27}$$
%$$f(x) = \frac{x^3 + x -1}{8 - x^3}$$
%$$g(x) = \frac{x^3 + 2x^2 -1}{6 - 2x^3}$$
%$$f(x) = \frac{x^2 + x+ 1}{1-x}$$
%$$g(x) = \frac{x-3}{2x^2 -1}$$
$$f(x) = \frac{x^2 + x + 1}{1+x}$$
$$g(x) = \frac{x-3}{2x^2 + 1}$$
\item Show, via the definition, that a function is or is not continuous at the given point.\\
%$x = 1$.
%$$f(x) = \frac{1}{x}$$
$$f(x) = \frac{x^2 + 5x + 6}{x + 3} \mbox{ when } x = 3$$
%$x = 0$.
%$$f(x) = \frac{x^2 + 5}{x+1}.$$
%not continuous when $x = 1$.
%$$f(x) = \left\{\begin{array}{rl}\frac{1}{x-1} &\mbox{if }x \neq 1\\ \pi &\mbox{if }x = 1\end{array}\right.$$
%$x = 1$.
%$$f(x) = x^2 + 4$$
%not when $x = 0$
%$$f{x} = \frac{5}{x^2 - 2x}$$
%$x = -2$ 
%$$f(x) = \frac{x+2}{x^2-4}$$

$$f(x) = \left\{\begin{array}{rl}2x-1 &\mbox{if }x \neq -3\\ 42 &\mbox{if }x = -3\end{array} \mbox{ when } x= - 3\right.$$
\item Apply the Intermediate Value Theorem to show a function has a zero on a given interval.\\
%$f(x) = 2^x - x^2$ has a real zero on the interval [0,3]
%$f(x) = x^3 + 4x -1$ has a real zero on the interval [0,1]
%$f(x) = \cos(x)$ has a real zero on the interval $[0,\pi]$
%$f(x) = x^5 + 13x -1$ has a real zero on the interval $[-1, 1]$
%$f(x) = 3x-5$ has a real zero on the interval $[0, 5]$.
%$g(x) = x-2$ has a real zero on the interval $[-7,7]$
$f(x) = \tan(x)$ has a real zero on the interval [-1/2, 1/2]\\
$g(x) = 5 - x^2$ has a real zero on the interval [0,3]
\item Calculate the derivative of a polynomial function directly from the definition.\\
%$$f(x) = 3x+5$$
%$$g(x) = 2 - 2x^2$$
%$$f(x) = 2x + 5$$
%$$g(x) = x^2 - 1$$
%$$f(x) = 6-4x$$
$$g(x) = 5 + 2x^2$$
%$$f(x) = x+3$$
$$g(x) = x^3  - 5$$
\item Calculate the derivative of a polynomial function using the power rule.\\
%Find $f'(x)$ $$f(x) = \frac{x + 5x^2}{x^3}$$
%$f'(x)$ $$f(x) = 5x^7 + \frac{6}{x^3}$$
%$$f(x) = 3x^3 + \sqrt{x}$$
%$$f(x) = \frac{3}{\sqrt{x^3}}$$
$$f(x) = \frac{x^2}{\sqrt{x}}$$
$$g(x) = 7x^{3/2}$$
%$$f(x) = 2x^{37}$$
\item Calculate the derivative of a trigonometric function.\\
%$f(x) = \sin(x)$
%$g(x) = \cos(x)$
%$f(x) = \tan(x)$
%$g(x) = \cot(x)$
%$f(x) = \sin(x)$
$$g(x) = \cos(x)$$
$$f(x) = \tan(x)$$
\item Calculate the derivative of a function using the product rule.\\ 
%$$f(x) = x^3\tan(x)$$
%$$f(x) = (3x^2 + 5)\tan(x)$$
$$f(x) = 2x^5\sin(x)$$
%$f(x) = (5x^5)(3x^3)$
%$g(x) = \sin(x)\cos(x)$
%$f(x) = \sin^2(x)$
$$g(x) = \tan(x)\sin(x)$$
\item Calculate the derivative of a function using the quotient rule.\\
%Find $f'(x)$ $$f(x) = \frac{\cos(x)}{2x^2 + 5x + 1}$$
%$$f(x) = \frac{\tan{x}}{x^4 + 2x^2 +1}$$
%$$f(x) = \frac{\sin(x)}{2x^2 +5}$$
%$g(x) = \frac{\sin(x)}{x^2}$
$$f(x) = \frac{5x^2}{\cos(x)}$$
$$g(x) = \frac{\cos(x)}{x^7}$$
%$f(x) = \frac{x^2 - 1}{\tan(x)}$
\item Calculate the derivative of a function using the chain rule.\\
%Find $f'(x)$ $$f(x) = \cos(x^4)$$
%Find $f'(x)$ $$f(x) = \sqrt{2x^4+ 2 -1}$$
%$$f(x) = \tan(3x^5)$$
$$f(x) = 3\cos^2(x)$$
%$$f(x) = \cos^2(x) - 5\cos(x) + 6$$
%$g(x) = \sin(\tan(x))$
$$f(x) = 5(2x^2 + x-1)^6$$
\item Calculate the derivative of a function using a combination of the power, product, quotient, and/or chain rule.\\
%$f(x) = 5(\sin(x)\cos(x))^2$
%$f(x) = \frac{\tan^2(x)}{3x^4 - 3x^2}$
%$g(x) = \frac{\sin(x^2)\cos(x)}{4x^5}$
$$f(x) = \left(\dfrac{4x^2}{\sin(x)}\right)^2$$
%$g(x) = (4x^5\cos(x))^5$
%$f(x) = (2x^3\sin(x)\cos(x))^4$
$$g(x) = \frac{\cos(x^2)}{\sin(x^3)}$$
\item Find the equation of a tangent line to a curve at a given point.\\
%Find the equation of the tangent line to the curve $y = x^3 - 2$ at the point $(1,-1)$.
%Find the equation of the tangent line to the curve $y = \frac{1}{x^2} - x$ at the point $(1,0)$.
%$y = 2x^2 + 4x -1$ at the point $(0,-1)$.
%$y = \sin(x) + 2x$ at the point $(0,0)$.
$y = \cos(\frac{x}{2})$ at the point $(2\pi,-1)$\\
$y = \tan^2(x)$ at the point $(\pi/4,1)$
\item Correctly find the derivative of an implicit function with respect to $x$.\\
$$x\sin(y) = x^2y$$
%$$x^3 + y^3 = 2xy$$
%$$\sin(xy) = \sqrt{xy}$$
%$$1 + xy = y^2 - x$$
$$y^2x - x^2y = xy$$
%$$\cos(xy) - y^2 - x^2 = 1$$
\item Set up a problem involving at least two related rates. To get this objective, find the equation that relates the changing variables, and correctly take the implicit derivative.\\
%A circle is increasing in area at 5 square unit per minute. How fast is the radius changing when the area is 25 square units?
%A 1.5m tall woman is walking towards a 10m tall lightpost at 2 m/s. How fast is the length of her shadow changing when she is 10m away from the post?
A cylindrical tank with radius $5m$ is being filled with water at a rate of $3m^3$/sec. How fast is the height of the water increasing? Recall that the volume of a cylinder is given by $V = \pi r^2 h$.\\
%Two cars start at the same spot. One drives due north at 55mph, and the other drives due west at 45mph. How fast is the difference between the two cars changing one hour after they start driving?
%The larger side of a rectangle is always twice as long as the smaller side. If the larger side is increasing a 2 inches per minute, how fast is the area increasing when the larger side is 12 inches long?
A $2m$ tall man is walking away from a $6m$ lightpost at $1.5m$ per second. How fast is the tip of his shadow moving away from the post when the man is $10m$ from the post?
\item Correctly solve the problems in item 16. Plug in the correct values into what you've set up, and find the solution to the given problem.
\item Identify the intervals on which the function is increasing and/or decreasing.\\
$$g(x) = 2x^3 + 4x^2 -5x + 4$$
$$f(x) = x^3 - 16x^2$$
\item Identify the intervals on which a function is concave up and/or concave down.\\
$$g(x) = 2x^3 + 4x^2 -5x + 4$$
$$f(x) = x^3 - 16x^2$$
\item Find all critical values of a function.\\
$$g(x) = 2x^3 + 4x^2 -5x + 4$$
$$f(x) = x^3 - 16x^2$$
\item \\
Use the first derivative test to classify the critical point $x = 3$ of $f(x) = 3x^2 - 18x + 5 $ as a local max, min, or neither.\\
Use the first derivative test to classify the critical points $x=5, -1$ of $f(x) = \frac{1}{3}x^3 - 2x^2 - 5x$ as a local max, min, or neither.
\item \\
Use the second derivative test to classify each of the critical point $x=1$ of $f(x) = e^x - ex + 1$ as a local max, min, or neither.\\
Use the second derivative test to find and classify the critical point $x=4$ of $f(x) = \frac{2}{3}(x-4)^{3/2}$ as a local max, min, or neither.
\item Apply the Extreme Value Theorem to a problem.\\
Find the $x$-values of the absolute maximum and absolute minimum of the function $f(x) =x^3 + 5x^2 - 5$ on the interval $[-1,1]$. State the hypotheses of any theorem you use.\\
Find the $x$-values of the absolute maximum and absolute minimum of the function $f(x) = x^3-2x^2-x+2$ on the interval $[-2,2]$. State the hypotheses of any theorem you use.
\item Apply the Mean Value Theorem to a problem.\\
Find the point in the interval [-2,2] that satisfies the Mean Value Theorem for the function $f(x) = 2x-x^2$. Verify that the appropriate hypotheses of the theorem are satisfied.\\
Find the point in the interval [0,1] that satisfies the Mean Value Theorem for the function $f(x) = x^2$. Verify that the appropriate hypotheses of the theorem are satisfied.
\item Correctly set up an optimization problem using the methods of Calculus. To get this objective, find the function that needs to be optimized, and use the constraint to reduce it to an equation of one variable.\\
A farmer has 130 feet of fencing, to enclose a rectangular pen. What is the largest area that can be enclosed?
If a cylinder needs to hold exactly 1500ml of liquid, what are the height and radius of the can that will minimize the surface area of metal used?\\
A farmer has 400 feet of fencing, to enclose a rectangular pen, with one side against a barn. What is the largest area that can be enclosed?
What is the volume of the largest rectangular box, with a square base, that can be made from 100 square feet of materials? The box will have 6 sides, so make sure you take all of that into account when computing the surface area.
\item Solve an optimization problem using the methods of Calculus. Find the appropriate local max or local min of the functions in item 25.\\
\item Calculate an antiderivative of a polynomial function.\\
$f(x) = \frac{2x^5 + 3x^4}{x^2}$\\
$g(x) = 3x^2 - 5x^4 + 1$
\item Calculate an antiderivative of a trigonometric function.\\
$f(x) = \cos(x) - \sin(x)$\\
$g(x) = \sec^2(x)$.
\item Use a finite summation to approximate the area under a curve.\\
Approximate the area under the curve $y = x^3$, over the interval [-1,1], with $n=4$. \\
Approximate the area under the curve $y = x^2 - 3x +1$, over the interval [0,2], with $n=3$. 
\item Write the formula for a Riemann sum to find the area under the curve. You do not need to evaluate the limit.\\
$f(x) = 2-2x$, over the interval $[1,7]$\\
$f(x) = x^4$, over the interval $[2,5]$
\item Evaluate a definite integral using the Fundamental Theorem of Calculus.\\
$$\int_{-1}^1 x^3 + 4x^2 - 6x + 1 dx$$
$$\int_{-5}^5 \sin(t) dt$$
\item Evaluate an indefinite integral.\\
$$\int e^w - \frac{1}{w} dw$$
$$ \int \sec^2(y) - \sec(y)\tan(y) dy$$
\item Evaluate an indefinite integral using substitution.\\
$$\int \sec^2(x) e^{\tan(x)}dx$$
$$\int 3t^2 \cos(t^3) dt$$
\item Calculate the derivative of a logarithmic function.\\
$f(x) = x\ln(x)$
$f(x) = (\ln(x))^2$
\item Calculate the derivative of an exponential function.\\
$$f(x) = e^x\cos(x)$$
$$f(x) = e^{\sin(x) + 7x}$$
\item Calculate a derivative using logarithmic differentiation.\\
$$y =x^x$$
$$y = [x^2 + 2x-1]^{x^2}$$
\item Calculate the derivative of an inverse trigonometric function.\\
$$f(x) = \tan^{-1}(x)$$
$$g(x) = \tan^{-1(\tan(x))}$$
\item Use L'Hopital's rule to evaluate a limit. Verify that the form is indeterminate.\\
$$\lim_{x\rightarrow 3} \frac{x-3}{27 - x^3}$$ 
$$\lim_{x\rightarrow 2} \frac{e^x - e^2}{x-2}$$
\end{enumerate}


\end{document}