\documentclass{article}
\usepackage[utf8]{inputenc}
\usepackage{fancyhdr}
\usepackage[margin=1.0in]{geometry}
\usepackage{amsmath}
\usepackage{changepage,stackrel}
\usepackage{graphicx}
\usepackage[inline]{enumitem}
\usepackage{enumitem}
\pagestyle{fancy}
\chead{Math 121 - Antiderivatives}

\begin{document}
\begin{enumerate}
\item For each of the functions below, compute the anti-derivative, and verify, via differentiation, that you are correct.
\begin{enumerate}
\item $x^2$
\item $3x^2$
\item $x^3$
\item $4x^3$
\item $x^4$
\item $1$
\item $0$
\item $\cos(x) - \sin(x)$
\item $\frac{1}{x}$
\item $x^{2/3}$
\item $4x^5 - 3x^7 + 11$
\item $\sec^2(x) + \frac{x^3}{4}$
\item $\frac{1}{2\sqrt{x}}$
\end{enumerate}
\item Make a conjecture about a general formula for the anti-derivative of the function $f(x) = x^n$. Use differentiation to show that you are right.
\item We can also use anti-derivatives to relate position and velocity. If the acceleration function for some particle is given by $a(t) = t^2 - 2$, and we know that both $v(0) = 5$ and $p(0) = 0$, find the formula for $p(t)$.
\end{enumerate}
\end{document}